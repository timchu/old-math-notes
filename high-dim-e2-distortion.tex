\title{Beating Indyk's $\sqrt{\log n}$ bound might be hard}
\documentclass[12pt]{article}
\usepackage{fullpage}
% \usepackage[backref=true]{biblatex}

\usepackage{amsmath,amssymb,amsthm}
\usepackage{graphicx}
\usepackage{hyperref}
\usepackage{float}

\title{$l_p$ to the $p$ power}

\author{
    Timothy Chu\\
        CMU\\
          \texttt{tzchu@cs.cmu.edu}
}

%\newcommand{\e}{\varepsilon}
%\newcommand{\len}{\ell}
%\newcommand{\R}{\mathbb{R}}
%\newcommand{\ourpath}{\mathrm{path}}
%\newcommand{\dist}{\mathbf{d}}
%\newcommand{\tri}{\triangle}
%\newcommand{\distto}{\mathbf{r}}
%\newcommand{\firsttet}{V_0V_1V_2V_3}
%\newcommand{\secondtet}{V'_0V'_1V'_2V'_3}
%\renewcommand{\because}[1]{&\left[\text{\small{#1}}\right]}
\newcommand{\blah}{hello \\ hello}

\newtheorem{problem}{Problem}
\newtheorem{theorem}{Theorem}[section]
\newtheorem{prop}[theorem]{Proposition}
\newtheorem{corollary}{Corollary}[theorem]
\newtheorem{remark}{Remark}[theorem]
\newtheorem{lemma}[theorem]{Lemma}
\newtheorem{definition}[theorem]{Definition}
\newtheorem{observation}[theorem]{Observation}


\usepackage{color}

\begin{document}
\maketitle


\begin{theorem} 
Euclidean distances $\subset$ Edge-squared distances.
\end{theorem}

\begin{proof}
If $X$ is the set of Euclidean distances on a finite set of points, then $\sqrt{X}$ is also a set of Euclidean distances (Schoenberg '37), where the square root is taken over all edge lengths.  Thus $\sqrt{X}^2 = X$ is a set of edge-squared distances. Therefore any set of Euclidean distances can be realized as a set of edge-squared distances.
\end{proof}

\begin{corollary} If there exists a configuration of points in Euclidean space such that any spanner of size $k$ has distortion $d$, there exists a configuraiton of points in Euclidean space such that any spanner of the edge-squared graph of size $k$ has distortion $d$.
\end{corollary}

Thus whatever lower bounds exist on Euclidean spanners (in arbitrary dimension), must be lower bounds on edge-squared spanners. Note that the Schoenberg embedding may require full dimensionality, so this argument will not hold for fixed dimension.

Thus improvement on Indyk's bound might be hard, as improving the bound for edge-squared also improves the bound for Euclidean metrics. Thus if Indyks' bound is lower-bound tight, then we can only hope to match it.

\end{document}
