\title{Default Title}
\documentclass[12pt]{article}
\usepackage{fullpage}
% \usepackage[backref=true]{biblatex}

\usepackage{amsmath,amssymb,amsthm}
\usepackage{graphicx}
\usepackage{hyperref}
\usepackage{float}

\title{$l_p$ to the $p$ power}

\author{
    Timothy Chu\\
        CMU\\
          \texttt{tzchu@cs.cmu.edu}
}

%\newcommand{\e}{\varepsilon}
%\newcommand{\len}{\ell}
%\newcommand{\R}{\mathbb{R}}
%\newcommand{\ourpath}{\mathrm{path}}
%\newcommand{\dist}{\mathbf{d}}
%\newcommand{\tri}{\triangle}
%\newcommand{\distto}{\mathbf{r}}
%\newcommand{\firsttet}{V_0V_1V_2V_3}
%\newcommand{\secondtet}{V'_0V'_1V'_2V'_3}
%\renewcommand{\because}[1]{&\left[\text{\small{#1}}\right]}
\newcommand{\blah}{hello \\ hello}

\newtheorem{problem}{Problem}
\newtheorem{theorem}{Theorem}[section]
\newtheorem{prop}[theorem]{Proposition}
\newtheorem{corollary}{Corollary}[theorem]
\newtheorem{remark}{Remark}[theorem]
\newtheorem{lemma}[theorem]{Lemma}
\newtheorem{definition}[theorem]{Definition}
\newtheorem{observation}[theorem]{Observation}


\usepackage{color}

\begin{document}
\maketitle


\section{Nearest Neighbor Charge of path not much larger than density-based path distance}
1. Assume uniform density. 
2. Draw a strip with small radius around the best path in the density-based metric.
3. Divide strip into cylindrical pieces such that with $1/2$ probability, $k=1$ points are in each pieces. The 'length' of the cylinder should be $p=2$ times its radius.
4. Paint a cylindrical piece black if there is a data point in it. For each point in the path, charge it to:
a) if the piece it's in is black, charge it to any point in the piece.
b) if the piece it's in is white, charge it to any point in the nearest black piece.
This forms an upper bound on the NN metric.
5. Now you can show that a) is a constant off from charging it to the midpoint of that piece (something like a 4 dilation, assuming $p=2$) and b) is a constant off from charging it to the midpoint of the nearest black piece. (with a < 2 dilation or something).
6.  Step 5 lets us charge to any point (e.g. the midpoint) of each black piece while picking up a small constant dilation. We can upper bound this charge by taking distance along the line rather than in Euclidean space.
7. Therefore, we can bound our .... crap  no we can't. Midpoints don't quite work. I just wanted to charge this to the 1D case, but it seems like things being close in the 1D case don't quite have a similar analog in the ND case.
8. We can, however, do the analysis where you lay down points in 1D line that's broken into regions, and 'snap' each point to the midpoint of each region.
9. We shoud get a very good concentration bound around this, but we can't actually do the 1D thing. Unless we take the 1D thing and add a fixed constant (roughly equal to the length of each piece).


\end{document}
