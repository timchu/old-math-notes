\title{Default Title}
\documentclass[12pt]{article}
\usepackage{fullpage}
% \usepackage[backref=true]{biblatex}

\usepackage{amsmath,amssymb,amsthm}
\usepackage{graphicx}
\usepackage{hyperref}
\usepackage{float}

\title{$l_p$ to the $p$ power}

\author{
    Timothy Chu\\
        CMU\\
          \texttt{tzchu@cs.cmu.edu}
}

%\newcommand{\e}{\varepsilon}
%\newcommand{\len}{\ell}
%\newcommand{\R}{\mathbb{R}}
%\newcommand{\ourpath}{\mathrm{path}}
%\newcommand{\dist}{\mathbf{d}}
%\newcommand{\tri}{\triangle}
%\newcommand{\distto}{\mathbf{r}}
%\newcommand{\firsttet}{V_0V_1V_2V_3}
%\newcommand{\secondtet}{V'_0V'_1V'_2V'_3}
%\renewcommand{\because}[1]{&\left[\text{\small{#1}}\right]}
\newcommand{\blah}{hello \\ hello}

\newtheorem{problem}{Problem}
\newtheorem{theorem}{Theorem}[section]
\newtheorem{prop}[theorem]{Proposition}
\newtheorem{corollary}{Corollary}[theorem]
\newtheorem{remark}{Remark}[theorem]
\newtheorem{lemma}[theorem]{Lemma}
\newtheorem{definition}[theorem]{Definition}
\newtheorem{observation}[theorem]{Observation}


\usepackage{color}

\begin{document}
\maketitle


\section{Edge-Squared Charge of path not much larger than density-based path distance}
\begin{enumerate}
\item Assume uniform density for now. Assume the path is continuous and smooth everywhere except a constant number of points. (Possibly a large assumption?)
\item Draw a strip of small radius around the best path in the density-based metric. Our strategy will be to find an edge-squared path that only uses points in this strip, that is not too far off from the density-based distance.

\begin{remark}
Our strategy is to reduce our general dimension problem to a $1$-dimensional one.
\end{remark}

\item Divide the strip into cylindrical-ish pieces, with radius $r$ and length $2r$. For small $r$, each piece can be made close to cylindrical. This will divide the strip into $N$ pieces. Sample enough points so that the strip has an expected value of $N$ points in the strip. (Note: working out concentration around here might be annoying. I haven't checked it out.)

\item The edge-squared path that we take will be: project each point in the strip onto the path. This induces an ordering on the points. Traverse the points in the given order.
\item Assume the length of the line is $1$. Then the distance between two adjacent points in the ordering is upper bounded by $\sqrt{(2r)^2 + s^2}$ where $s$ is the distance between their projections \textit{along the path}. (Ish.)

\item Note that the expected value of $s$ is $r$.  Note that $N$ is roughly $1/r$, or that $r$ is roughly $1/N$. (This is if we assume the path length is 1. Here is where I've gotten fuzzy, as it's pretty loud around me right now....)

\item Therefore, this problem is equivalent to laying down $N$ points on the line (so expected spacing $1/N$), taking adjacent distances, squaring the result, and adding $2r^2 \cdot N = 2(1/N)^2 \cdot N = 2/N$ which is less than twice the expected value of the 1-dimensional problem (assuming equal spacing, the edge-squared cost in the 1-dimensional problem is $1/N$. Non-equal spacing onlyi increases the cost.)

Therfore we have upper bounded our edge-squared distance with the edge-squared distance of the $1$-dimensional problem, times $3$. Ish. (Here, I assumed some concentration of expected number of points in the strip). I'm sure you can do better (you can probably do a 1-approximation thing, maybe with the $NN$ distance), but eh.


\end{enumerate}


\end{document}
